\documentclass[conference]{IEEEtran}

\IEEEoverridecommandlockouts
\usepackage{cite}
\usepackage{amsmath,amssymb,amsfonts}
\usepackage{algorithmic}
\usepackage{graphicx}
\usepackage{textcomp}
\usepackage{xcolor}
\usepackage{fancyhdr}
\usepackage{lipsum}

\def\BibTeX{{\rm B\kern-.05em{\sc i\kern-.025em b}\kern-.08em
    T\kern-.1667em\lower.7ex\hbox{E}\kern-.125emX}}
    
\fancypagestyle{firstpagefooter}{%
  \fancyhf{}
  \renewcommand\headrulewidth{0pt}
  \fancyfoot[R]{Drone Application in Precision Agriculture, Hamm-Lippstadt Hochschule}
}

\pagestyle{empty}

\bibliographystyle{IEEEtran}

\begin{document}
\title{Drone Application in Precision Agriculture}

\author{\IEEEauthorblockN{Vytaras Juraska}
\IEEEauthorblockA{\textit{Electronics Engineering (7\textsuperscript{th} Semester)} \\
\textit{Hamm-Lippstadt Hochschule}\\
Lippstadt, Germany \\
vytaras.juraska@stud.hshl.de}
}

\maketitle

\begin{abstract}
overview of the concept of precision agriculture, its usual methods and appliances, including the current state of drone adaptation to the application of precision agriculture, comparing the current market and analysing the future of drone use in this field of adaptation.
\end{abstract}

\thispagestyle{firstpagefooter}

\section{Introduction}

    Over the years, a lot of burning questions come up, which deducts how well and for how long the human race can survive for on the land, that we've got. There are many, still unsolved and exponentially growing issues, which we still have no solution to, but one of the biggest topics, which will most likely always be an important topic upon which we, as a humankind can improve - agriculture.
    
    It is self explanatory, more people on earth, requires more food, more food - use more land. But land is not an endless resource, so what else can we do, to keep growing the quantity and quality of the food production, keeping in mind, that land is a precious resource, which we can not use for granted. Well this specific topic is expanded and still remains to be a very important subject 'till this day, called Precision Agriculture.

\section{Precision Agriculture}

    Introduction and quick overview of precision agriculture.

\subsection{Technology}

    What kind of technology is commonly used and what kind of methods are usually applied in using the technology efficiently.
    
\subsection{Variables}

    What kind of variables are commonly collected, most required and prioritised.
    
\subsection{Implementation}

    Common implementation and practices used to get the best outcome.
    
\section{Drone Development}

    Quick introduction and overview to a drone.

\subsection{Common Fields of Adaptation}

    Description of which fields drones are used, what kind of modifications are usually applied for the drones to adapt to the field of application.
    
\section{Drone Application to Precision Agriculture}

    Quick idea and overview of how it is being implemented into this field.
    
\subsection{Current Market and Features}

    Mention and describe a couple of currently existing drones for precision agriculture, what features they provide and what is the price difference (if applicable).

\section{Advantages of Drone Application to Precision Agriculture}

    Mention and pick out a couple of main points why drones should be used in Precision Agriculture

\section{Disadvantages of Drone Application to Precision Agriculture}

    Which fields other methods might give better or quicker results.

\section{Summary and Conclusion}

    Summary of paper and conclusion to the whole topic in one.

\section{Affidavit}
I, Vytaras Juraska, herewith declare that I have composed the present paper and work by our self and without use of any other than the cited sources and aids. Sentences or parts of sentences quoted literally are marked as such; other references with regard to the statement and scope are indicated by full details of the publications concerned. The paper and work in the same or similar form has not been submitted to any examination body and has not been published. This paper was not yet, even in part, used in another examination or as a course performance.

\end{document}
